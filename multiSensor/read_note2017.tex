 \documentclass[a4paper, 11pt]{article}

%%%%%% 导入包 %%%%%%
\usepackage{CJKutf8}
\usepackage{graphicx}
\usepackage[unicode]{hyperref}
\usepackage{xcolor}
\usepackage{color}
\usepackage{cite}
\usepackage{indentfirst}
\usepackage{tikz,mathpazo}
\usetikzlibrary{shapes.geometric, arrows}
%%%%%% 设置字号 %%%%%%
\newcommand{\chuhao}{\fontsize{42pt}{\baselineskip}\selectfont}
\newcommand{\xiaochuhao}{\fontsize{36pt}{\baselineskip}\selectfont}
\newcommand{\yihao}{\fontsize{28pt}{\baselineskip}\selectfont}
\newcommand{\erhao}{\fontsize{21pt}{\baselineskip}\selectfont}
\newcommand{\xiaoerhao}{\fontsize{18pt}{\baselineskip}\selectfont}
\newcommand{\sanhao}{\fontsize{15.75pt}{\baselineskip}\selectfont}
\newcommand{\sihao}{\fontsize{14pt}{\baselineskip}\selectfont}
\newcommand{\xiaosihao}{\fontsize{12pt}{\baselineskip}\selectfont}
\newcommand{\wuhao}{\fontsize{10.5pt}{\baselineskip}\selectfont}
\newcommand{\xiaowuhao}{\fontsize{9pt}{\baselineskip}\selectfont}
\newcommand{\liuhao}{\fontsize{7.875pt}{\baselineskip}\selectfont}
\newcommand{\qihao}{\fontsize{5.25pt}{\baselineskip}\selectfont}

%%%% 设置 section 属性 %%%%
\makeatletter
\renewcommand\section{\@startsection{section}{1}{\z@}%
{-1.5ex \@plus -.5ex \@minus -.2ex}%
{.5ex \@plus .1ex}%
{\normalfont\sihao\CJKfamily{hei}}}
\makeatother

%%%% 设置 subsection 属性 %%%%
\makeatletter
\renewcommand\subsection{\@startsection{subsection}{1}{\z@}%
{-1.25ex \@plus -.5ex \@minus -.2ex}%
{.4ex \@plus .1ex}%
{\normalfont\xiaosihao\CJKfamily{hei}}}
\makeatother

%%%% 设置 subsubsection 属性 %%%%
\makeatletter
\renewcommand\subsubsection{\@startsection{subsubsection}{1}{\z@}%
{-1ex \@plus -.5ex \@minus -.2ex}%
{.3ex \@plus .1ex}%
{\normalfont\xiaosihao\CJKfamily{hei}}}
\makeatother

%%%% 段落首行缩进两个字 %%%%
\makeatletter
\let\@afterindentfalse\@afterindenttrue
\@afterindenttrue
\makeatother
\setlength{\parindent}{2em}  %中文缩进两个汉字位


%%%% 下面的命令重定义页面边距,使其符合中文刊物习惯 %%%%
\addtolength{\topmargin}{-54pt}
\setlength{\oddsidemargin}{0.63cm}  % 3.17cm - 1 inch
\setlength{\evensidemargin}{\oddsidemargin}
\setlength{\textwidth}{14.66cm}
\setlength{\textheight}{24.00cm}    % 24.62

%%%% 下面的命令设置行间距与段落间距 %%%%
\linespread{1.4}
% \setlength{\parskip}{1ex}
\setlength{\parskip}{0.5\baselineskip}

%%%% 正文开始 %%%%
\begin{document}
\begin{CJK}{UTF8}{gbsn}

%%%% 定理类环境的定义 %%%%
\newtheorem{example}{例}             % 整体编号
\newtheorem{algorithm}{算法}
\newtheorem{theorem}{定理}[section]  % 按 section 编号
\newtheorem{definition}{定义}
\newtheorem{axiom}{公理}
\newtheorem{property}{性质}
\newtheorem{proposition}{命题}
\newtheorem{lemma}{引理}
\newtheorem{corollary}{推论}
\newtheorem{remark}{注解}
\newtheorem{condition}{条件}
\newtheorem{conclusion}{结论}
\newtheorem{assumption}{假设}




%%%% 定义标题格式,包括title,author,affiliation,email等 %%%%
\title{ 阅读论文综述}
\author{王俊杰\footnote{电子邮件: wangjunjie2013@gmail.com}\\[2ex]
\xiaosihao 哈尔滨工业大学\\[2ex]
}
\date{}


%%%% 以下部分是正文 %%%%
\maketitle


 \begin{tabular}{|c|ccccccccccc|}
\hline
正体&$\Gamma$ & $\Delta$ & $\Theta$ & $\Lambda$ & $\Xi$ & $\Pi$ & $\Sigma$ & $\Upsilon$ & $\Phi$ & $\Psi$ & $\Omega$\\
\hline
\verb|\mit|斜体&$\mit\Gamma$ & $\mit\Delta$ & $\mit\Theta$ & $\mit\Lambda$ & $\mit\Xi$ & $\mit\Pi$ & $\mit\Sigma$ &  $\mit\Upsilon$ & $\mit\Phi$ & $\mit\Psi$ & $\mit\Omega$\\
\hline
\end{tabular}


 \begin{tabular}{|lcc|lcc|}
\hline
命令 & 大写 & 小写 & 命令 & 大写 & 小写 \\
\hline
  alpha & $A$ & $\alpha$ &  beta & $B$ &$\beta$  \\
  gamma & $\Gamma$ & $\gamma$  &  delta & $\Delta$ & $\delta$ \\
  epsilon & $E$ & $\epsilon,\varepsilon$ &  zeta & $Z$ & $\zeta$ \\
   eta & $H$ &$\eta$  &  theta & $\Theta$ & $\theta,\vartheta$ \\
  iota & $I$ & $\iota$ &   kappa & $K$ & $\kappa$ \\
  lambda & $\Lambda$ & $\lambda$  & mu & $M$ & $\mu$ \\
  nu & $N$ & $\nu$ & omicron & $O$ & $o$ \\
    xi & $\Xi$ & $\xi$  &   pi & $\Pi$ & $\pi,\varpi$ \\
    rho & $P$ & $\rho,\varrho$  &  sigma & $\Sigma$ & $\sigma,\varsigma$ \\
   tau & $T$ & $\tau$ &   upsilon & $\Upsilon$ & $\upsilon$ \\
  phi & $\Phi$ & $\phi,\varphi$ &  chi & $X$ & $\chi$ \\
  psi & $\Psi$ & $\psi$  &  omega & $\Omega$ &$\omega$ \\
\hline
\end{tabular}


\newpage
\tableofcontents
\newpage
\section{On CPHD Filters With Track Labeling}
\subsection{The distribution and p.g.fl.'s LRFS's}
In LRFS theory, the state of a target has the form $(x,l)$, where $x$ is an element of some region of a Euclidean space, and where $l$ is an element of a countable set of distince labels.

Given this, a multitarget state is a finite set
\begin{equation}
X = \left\{(x_1,l_1),...,(x_n,l_n) \right\}
\end{equation}
If $f(x,l)$ is a function of the variable $x = (x,l)$ which is integrable with respect to $x$ and is such that $f(x,l) = 0$ identically for all but a finite number of $l \in L$, then its integral is
\begin{equation}
\int f(x)dx = \sum_{l \in L} \int f(x,l)dx
\end{equation}
The set integral of a function $f(X)$ of a finite-set variable $X$ is
\begin{equation}
\int f(X)\delta X = \sum_{n \geq 0} \frac{1}{n!} \int f(x_1,...x_n)dx_1...,x_n = \sum_{n \geq 0} \frac{1}{n!} \sum_{l_1,...l_n \in L^n}f( (x_1,l_1),...(x_n,l_n)dx_1...,x_n
\end{equation}

 \subsection{GLMB distribution}
 A probability distribution $f(X)$ is a GLMB distribution if it has the following form:
 \begin{equation}
f(X) = \delta \sum_{o \in O} w_o (X) s_o^X
 \end{equation}
where $O$ is a finite set of indices $o$.$s_{o,l}(x) = s_o(x,l)$ is the track distribution corresponding to the track label $l$. The p.g.fl. of a GLMB distribution is
\begin{equation}
G[h] = \sum_{o \in O} \sum_{L \in \mathcal{L}}w_o(L) \prod_{l \in L} s_{o,l} [h]
\end{equation}

A LMB distribution is a mono-GLMB distribution such that $w(L)$ has the form
\begin{equation}
w^J(L) = \prod_{l \in J - L}(1 - q_l) \prod_{l\ in L} (q_l \textbf{1}_L(l))
\end{equation}
Its p.g.fl is
\begin{equation}
G[h] = \prod_{l \in J} (1 - q_l + q_l * sl[h])
\end{equation}
\subsection{Labeled PHD filter}
\subsubsection{LPHD prediction}
Let the target-appearance and posterior p.g.fl.'s at time $k$ be, respectively,
\begin{equation}
G_{k+1|k}^B[h] = \prod_{l \in L_B}(1 - q_{l,B}^{k+1} + q_{l,B}^{k+1}s_{l,B}^{k+1}[h])
\end{equation}
\begin{equation}
G_{k|k}[h] = \prod_{l \in L_{k|k}}(1 - q_l^{k|k} + q_{l}^{k|k}s_{l}^{k|k}[h])
\end{equation}
Then the time updated p.g.fl. is
\begin{equation}
G_{k+1|k}[h] = \prod_{l \in L_{k+1|k}} (1 - q_l^{k+1|k} + 1_l^{k+1|k} s_l^{k+1|k})[h]
\end{equation}
\subsubsection{LPHD update}




\section{Association-Free Direct Filtering of Multi-Target Random Finite Sets with Set Distance Measures}
For recursive Bayesian filtering, we now face two problems:
\begin{itemize}
\item Prior density and measurement density are given by collection of particles only
\item The assoication of target between particle densities is unknown
\end{itemize}
Traditional MTT algorithms employ a joint state vector that contains the individual target states and aimt at minimizing the MSE of the estimate.
\subsection{Distance Measure}


\section{Artificial neural network training utilizing the smooth variable
structure filter estimation strategy}

\subsection{Traditional ANN training techniques}
BP is one of the first used techniques in training of multilayer perceptrons. BP is a first order stochastic gradient decent method that iteratively searches for link weights that minimize the output error in a supervised manner.  Quasi-Newton method demonstrated better convergence performance than the standard BP algorithm, but it requres large memory storage to store the Hessian matrix.
\subsection{State estimation-based ANN training}
A new hybird learning algorithm that comines the EKF and particle filter has been presented. The new training scheme provides faster speed of convergence than the stand-alone EKF.
\subsection{Feed-forward multilayered neural network}
The operation of node $(n+1,i)$ is described by the following eqution
\begin{equation}
x_i^{n+1}(t) = \varphi (\sum_{j=1}^{N_n -1}w_{i,j}^nx_j^n(t) + b_i^{n+1})
\end{equation}
\subsection{Global and decoupled EKF-based NN training}
The EKF has been tailored to train feed-forward neural networks by formulating the network as a filtering problem. Accordingly, feed-forward multilayer perception network behavior can be described by a nonlinear discrete time state space representation such that
\begin{equation}
w_{k+1} = w_k + \omega_k
\end{equation}
\begin{equation}
y_k =  C_k(w_k,u_k) + v_k
\end{equation}
It demonstrates the neural network as a stationary system with an additional zero-mean, white system noise $\omega_k$ with a covariance described by $[\omega_k \omega_l^T] = \delta_{k,l}Q_k$. Neural network weights and biases $w_k$ are regarded as the system's state. Meausrement function is a nonlinear equation relating network desired response $y_k$ to the network input $u_k$ and weights $w_k$.

\section{DNN and swithching kalman filter based continuous affect recognition}
The DD-SKF framework firstly models the complex nonlinear relationship between the input features and the affective dimensions via the non-recurrent DNN, then models the temporal dynamics embedded in the emotions via the segmental linear SKF.



\section{The Adaptive LMB filter}
This paper proposes a new multi-Bernoulli filter called the Adaptive
LMB filter. It combines the relative strengths of the known $\delta$-GLMB
and the LMB filter.

\subsection{introduction}
The aim of MTT is the estimation of the number of objects as well as their
individual states based on noisy measurements, where missed detections and false
alarms lead to ambiguities in the track-to-measurement association.

The adaptive LMB filter is proposed which automatically switches between
an LMB and $\delta$-GLMB representation based on KL divergence and
entropy.

\subsection{LMB RFS}
In a multi-object scenario, it is often required to estimate the identity
of an object in addition to its current state. A labeled RFS is an RFS
on $X \times L$ with state space $X$ and finite label space $L$. A LMB RFS is
completely defined by the parameter $\left\{(r^l,p^l) \right\}_{l \in L}$
and its density is given by
\begin{equation}
\bold{\pi}(X) = \Delta(X)w(L(x))p^X
\end{equation}
where
\begin{equation}
w(L) = \prod_{i \in L}(1 - r^i) \prod_{l\in L} \frac{1_L(l)r^l}{1 - r^l}
\end{equation}

\subsection{switch criteria}
The aim of the ALMB filter is to switch automatically between the LMB approximation,
facilitating a fast propogation of the density, and the more accurate
$\delta$-GLMB density.
\subsubsection{KL criterion}
It is shown that the LMB approximation loses information about the
cardinality distribution. Thus, an intuitive way to detect the information loss, is
to examine the difference between the posterior cardinality distribution of the
$\delta$-GLMB RFS and its LMB approximation.

\subsubsection{Entropy Criterion}
The KL criterion does not detect challenging situations with ambiguous
data association if the cardinality distributions are identical. Hence,
a measure for the data association uncertainty is required which enables
a switching to the more accurate $\delta$-GLMB representation in these
situations.

\section{Average Marginal Density Based Distributed Multichannel Fusion for Multi-target Tracking}


\section{Distributed localisation of sensors with partially overlapping field-of-views in fusion networks}
\subsection{Abstract}
Each sensor has a partially overlapping FoV with its neighbours, and, collects both target originated and spurious measurements. We are interested in estimating the locations of the sensors in a network coordinate system using only these measurements.


\section{Multiple Object Tracking in Unknown Backgrounds with Labeled Random Finite Sets}






\end{CJK}
\end{document}
 \documentclass[a4paper, 11pt]{article}

%%%%%% 导入包 %%%%%%
\usepackage{CJKutf8}
\usepackage{graphicx}
\usepackage[unicode]{hyperref}
\usepackage{xcolor}
\usepackage{color}
\usepackage{cite}
\usepackage{indentfirst}
\usepackage{tikz,mathpazo}
\usetikzlibrary{shapes.geometric, arrows}
%%%%%% 设置字号 %%%%%%
\newcommand{\chuhao}{\fontsize{42pt}{\baselineskip}\selectfont}
\newcommand{\xiaochuhao}{\fontsize{36pt}{\baselineskip}\selectfont}
\newcommand{\yihao}{\fontsize{28pt}{\baselineskip}\selectfont}
\newcommand{\erhao}{\fontsize{21pt}{\baselineskip}\selectfont}
\newcommand{\xiaoerhao}{\fontsize{18pt}{\baselineskip}\selectfont}
\newcommand{\sanhao}{\fontsize{15.75pt}{\baselineskip}\selectfont}
\newcommand{\sihao}{\fontsize{14pt}{\baselineskip}\selectfont}
\newcommand{\xiaosihao}{\fontsize{12pt}{\baselineskip}\selectfont}
\newcommand{\wuhao}{\fontsize{10.5pt}{\baselineskip}\selectfont}
\newcommand{\xiaowuhao}{\fontsize{9pt}{\baselineskip}\selectfont}
\newcommand{\liuhao}{\fontsize{7.875pt}{\baselineskip}\selectfont}
\newcommand{\qihao}{\fontsize{5.25pt}{\baselineskip}\selectfont}

%%%% 设置 section 属性 %%%%
\makeatletter
\renewcommand\section{\@startsection{section}{1}{\z@}%
{-1.5ex \@plus -.5ex \@minus -.2ex}%
{.5ex \@plus .1ex}%
{\normalfont\sihao\CJKfamily{hei}}}
\makeatother

%%%% 设置 subsection 属性 %%%%
\makeatletter
\renewcommand\subsection{\@startsection{subsection}{1}{\z@}%
{-1.25ex \@plus -.5ex \@minus -.2ex}%
{.4ex \@plus .1ex}%
{\normalfont\xiaosihao\CJKfamily{hei}}}
\makeatother

%%%% 设置 subsubsection 属性 %%%%
\makeatletter
\renewcommand\subsubsection{\@startsection{subsubsection}{1}{\z@}%
{-1ex \@plus -.5ex \@minus -.2ex}%
{.3ex \@plus .1ex}%
{\normalfont\xiaosihao\CJKfamily{hei}}}
\makeatother

%%%% 段落首行缩进两个字 %%%%
\makeatletter
\let\@afterindentfalse\@afterindenttrue
\@afterindenttrue
\makeatother
\setlength{\parindent}{2em}  %中文缩进两个汉字位


%%%% 下面的命令重定义页面边距,使其符合中文刊物习惯 %%%%
\addtolength{\topmargin}{-54pt}
\setlength{\oddsidemargin}{0.63cm}  % 3.17cm - 1 inch
\setlength{\evensidemargin}{\oddsidemargin}
\setlength{\textwidth}{14.66cm}
\setlength{\textheight}{24.00cm}    % 24.62

%%%% 下面的命令设置行间距与段落间距 %%%%
\linespread{1.4}
% \setlength{\parskip}{1ex}
\setlength{\parskip}{0.5\baselineskip}

%%%% 正文开始 %%%%
\begin{document}
\begin{CJK}{UTF8}{gbsn}

%%%% 定理类环境的定义 %%%%
\newtheorem{example}{例}             % 整体编号
\newtheorem{algorithm}{算法}
\newtheorem{theorem}{定理}[section]  % 按 section 编号
\newtheorem{definition}{定义}
\newtheorem{axiom}{公理}
\newtheorem{property}{性质}
\newtheorem{proposition}{命题}
\newtheorem{lemma}{引理}
\newtheorem{corollary}{推论}
\newtheorem{remark}{注解}
\newtheorem{condition}{条件}
\newtheorem{conclusion}{结论}
\newtheorem{assumption}{假设}

%%%% 重定义 %%%%
\renewcommand{\contentsname}{目录}  % 将Contents改为目录
\renewcommand{\abstractname}{摘要}  % 将Abstract改为摘要
\renewcommand{\refname}{参考文献}   % 将References改为参考文献
\renewcommand{\indexname}{索引}
\renewcommand{\figurename}{图}
\renewcommand{\tablename}{表}
\renewcommand{\appendixname}{附录}
\renewcommand{\algorithm}{算法}


%%%% 定义标题格式,包括title,author,affiliation,email等 %%%%
\title{ 阅读论文综述}
\author{王俊杰\footnote{电子邮件: wangjunjie2013@gmail.com}\\[2ex]
\xiaosihao 哈尔滨工业大学\\[2ex]
}
\date{}


%%%% 以下部分是正文 %%%%
\maketitle


 \begin{tabular}{|c|ccccccccccc|}
\hline
正体&$\Gamma$ & $\Delta$ & $\Theta$ & $\Lambda$ & $\Xi$ & $\Pi$ & $\Sigma$ & $\Upsilon$ & $\Phi$ & $\Psi$ & $\Omega$\\
\hline
\verb|\mit|斜体&$\mit\Gamma$ & $\mit\Delta$ & $\mit\Theta$ & $\mit\Lambda$ & $\mit\Xi$ & $\mit\Pi$ & $\mit\Sigma$ &  $\mit\Upsilon$ & $\mit\Phi$ & $\mit\Psi$ & $\mit\Omega$\\
\hline
\end{tabular}


 \begin{tabular}{|lcc|lcc|}
\hline
命令 & 大写 & 小写 & 命令 & 大写 & 小写 \\
\hline
  alpha & $A$ & $\alpha$ &  beta & $B$ &$\beta$  \\
  gamma & $\Gamma$ & $\gamma$  &  delta & $\Delta$ & $\delta$ \\
  epsilon & $E$ & $\epsilon,\varepsilon$ &  zeta & $Z$ & $\zeta$ \\
   eta & $H$ &$\eta$  &  theta & $\Theta$ & $\theta,\vartheta$ \\
  iota & $I$ & $\iota$ &   kappa & $K$ & $\kappa$ \\
  lambda & $\Lambda$ & $\lambda$  & mu & $M$ & $\mu$ \\
  nu & $N$ & $\nu$ & omicron & $O$ & $o$ \\
    xi & $\Xi$ & $\xi$  &   pi & $\Pi$ & $\pi,\varpi$ \\
    rho & $P$ & $\rho,\varrho$  &  sigma & $\Sigma$ & $\sigma,\varsigma$ \\
   tau & $T$ & $\tau$ &   upsilon & $\Upsilon$ & $\upsilon$ \\
  phi & $\Phi$ & $\phi,\varphi$ &  chi & $X$ & $\chi$ \\
  psi & $\Psi$ & $\psi$  &  omega & $\Omega$ &$\omega$ \\
\hline
\end{tabular}


\newpage
\section{Inference via low-dimensional couplings}
The transport map $T$ can be viewed as a transormation that moves particles : given a collection of samples from $v_{\eta}$ , $T$ rearranges them in accordance with the new distribution $v_{\pi}$

Optimal transprt maps, for instance, define couplings that minimize a particular integrated transport cost expressing the effort required to rearrage samples. In recent years, several other couplings have been proposed for use in statistical problems, e.g.,

\begin{itemize}
\item parametric approximations -- Moselhy, T. and Marzouk, Y. (2012). Bayesian inference with optimal maps. Journal of Computational Physics 231 7815–7850.
\item Knote-Rosenblatt rearrangement-- Rosenblatt, M. (1952). Remarks on a multivariate transformation. The Annals of Mathematical Statistics 470–472
\item coupling induced by ODE flows--- Heng, J., Doucet, A. and Pokern, Y. (2015). Gibbs flow for approximate transport with applications to Bayesian computation. arXiv:1509.08787.
Daum, F. and Huang, J. (2008). Particle flow for nonlinear filters with log-homotopy. In SPIE Defense and Security Symposium 696918–696918. International Society for Optics and Photonics.
Anderes, E. and Coram, M. (2012). A general spline representation for nonparametric and semiparametric density estimates using diffeomorphisms. arXiv:1205.5314.
\end{itemize}

\textbf{Yet the construction, representation, and evaluation of all these maps grows challenging in high dimensions.}

The central contribution of this paper is to \textbf{establish a link between the conditional independence structure of the target measure and the existence of special low dimensional coupling. These couplings are induced by transport maps that are sparse or decomposable.}
\end{CJK}
\end{document}
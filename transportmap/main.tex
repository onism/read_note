 \documentclass[a4paper, 11pt]{article}

%%%%%% 导入包 %%%%%%
\usepackage{CJKutf8}
\usepackage{graphicx}
\usepackage[unicode]{hyperref}
\usepackage{xcolor}
\usepackage{color}
\usepackage{cite}
\usepackage{indentfirst}
\usepackage{tikz,mathpazo}
\usepackage{amsmath}
\usepackage{amsfonts}
\usetikzlibrary{shapes.geometric, arrows}
%%%%%% 设置字号 %%%%%%
\newcommand{\chuhao}{\fontsize{42pt}{\baselineskip}\selectfont}
\newcommand{\xiaochuhao}{\fontsize{36pt}{\baselineskip}\selectfont}
\newcommand{\yihao}{\fontsize{28pt}{\baselineskip}\selectfont}
\newcommand{\erhao}{\fontsize{21pt}{\baselineskip}\selectfont}
\newcommand{\xiaoerhao}{\fontsize{18pt}{\baselineskip}\selectfont}
\newcommand{\sanhao}{\fontsize{15.75pt}{\baselineskip}\selectfont}
\newcommand{\sihao}{\fontsize{14pt}{\baselineskip}\selectfont}
\newcommand{\xiaosihao}{\fontsize{12pt}{\baselineskip}\selectfont}
\newcommand{\wuhao}{\fontsize{10.5pt}{\baselineskip}\selectfont}
\newcommand{\xiaowuhao}{\fontsize{9pt}{\baselineskip}\selectfont}
\newcommand{\liuhao}{\fontsize{7.875pt}{\baselineskip}\selectfont}
\newcommand{\qihao}{\fontsize{5.25pt}{\baselineskip}\selectfont}

%%%% 设置 section 属性 %%%%
\makeatletter
\renewcommand\section{\@startsection{section}{1}{\z@}%
{-1.5ex \@plus -.5ex \@minus -.2ex}%
{.5ex \@plus .1ex}%
{\normalfont\sihao\CJKfamily{hei}}}
\makeatother

%%%% 设置 subsection 属性 %%%%
\makeatletter
\renewcommand\subsection{\@startsection{subsection}{1}{\z@}%
{-1.25ex \@plus -.5ex \@minus -.2ex}%
{.4ex \@plus .1ex}%
{\normalfont\xiaosihao\CJKfamily{hei}}}
\makeatother

%%%% 设置 subsubsection 属性 %%%%
\makeatletter
\renewcommand\subsubsection{\@startsection{subsubsection}{1}{\z@}%
{-1ex \@plus -.5ex \@minus -.2ex}%
{.3ex \@plus .1ex}%
{\normalfont\xiaosihao\CJKfamily{hei}}}
\makeatother

%%%% 段落首行缩进两个字 %%%%
\makeatletter
\let\@afterindentfalse\@afterindenttrue
\@afterindenttrue
\makeatother
\setlength{\parindent}{2em}  %中文缩进两个汉字位


%%%% 下面的命令重定义页面边距,使其符合中文刊物习惯 %%%%
\addtolength{\topmargin}{-54pt}
\setlength{\oddsidemargin}{0.63cm}  % 3.17cm - 1 inch
\setlength{\evensidemargin}{\oddsidemargin}
\setlength{\textwidth}{14.66cm}
\setlength{\textheight}{24.00cm}    % 24.62

%%%% 下面的命令设置行间距与段落间距 %%%%
\linespread{1.4}
% \setlength{\parskip}{1ex}
\setlength{\parskip}{0.5\baselineskip}

%%%% 正文开始 %%%%
\begin{document}
\begin{CJK}{UTF8}{gbsn}

%%%% 定理类环境的定义 %%%%
\newtheorem{example}{例}             % 整体编号
\newtheorem{algorithm}{算法}
\newtheorem{theorem}{定理}[section]  % 按 section 编号
\newtheorem{definition}{定义}
\newtheorem{axiom}{公理}
\newtheorem{property}{性质}
\newtheorem{proposition}{命题}
\newtheorem{lemma}{引理}
\newtheorem{corollary}{推论}
\newtheorem{remark}{注解}
\newtheorem{condition}{条件}
\newtheorem{conclusion}{结论}
\newtheorem{assumption}{假设}

%%%% 重定义 %%%%
\renewcommand{\contentsname}{目录}  % 将Contents改为目录
\renewcommand{\abstractname}{摘要}  % 将Abstract改为摘要
\renewcommand{\refname}{参考文献}   % 将References改为参考文献
\renewcommand{\indexname}{索引}
\renewcommand{\figurename}{图}
\renewcommand{\tablename}{表}
\renewcommand{\appendixname}{附录}
\renewcommand{\algorithm}{算法}


%%%% 定义标题格式,包括title,author,affiliation,email等 %%%%
\title{ 阅读论文综述}
\author{王俊杰\footnote{电子邮件: wangjunjie2013@gmail.com}\\[2ex]
\xiaosihao 哈尔滨工业大学\\[2ex]
}
\date{}


%%%% 以下部分是正文 %%%%
\maketitle


 \begin{tabular}{|c|ccccccccccc|}
\hline
正体&$\Gamma$ & $\Delta$ & $\Theta$ & $\Lambda$ & $\Xi$ & $\Pi$ & $\Sigma$ & $\Upsilon$ & $\Phi$ & $\Psi$ & $\Omega$\\
\hline
\verb|\mit|斜体&$\mit\Gamma$ & $\mit\Delta$ & $\mit\Theta$ & $\mit\Lambda$ & $\mit\Xi$ & $\mit\Pi$ & $\mit\Sigma$ &  $\mit\Upsilon$ & $\mit\Phi$ & $\mit\Psi$ & $\mit\Omega$\\
\hline
\end{tabular}


 \begin{tabular}{|lcc|lcc|}
\hline
命令 & 大写 & 小写 & 命令 & 大写 & 小写 \\
\hline
  alpha & $A$ & $\alpha$ &  beta & $B$ &$\beta$  \\
  gamma & $\Gamma$ & $\gamma$  &  delta & $\Delta$ & $\delta$ \\
  epsilon & $E$ & $\epsilon,\varepsilon$ &  zeta & $Z$ & $\zeta$ \\
   eta & $H$ &$\eta$  &  theta & $\Theta$ & $\theta,\vartheta$ \\
  iota & $I$ & $\iota$ &   kappa & $K$ & $\kappa$ \\
  lambda & $\Lambda$ & $\lambda$  & mu & $M$ & $\mu$ \\
  nu & $N$ & $\nu$ & omicron & $O$ & $o$ \\
    xi & $\Xi$ & $\xi$  &   pi & $\Pi$ & $\pi,\varpi$ \\
    rho & $P$ & $\rho,\varrho$  &  sigma & $\Sigma$ & $\sigma,\varsigma$ \\
   tau & $T$ & $\tau$ &   upsilon & $\Upsilon$ & $\upsilon$ \\
  phi & $\Phi$ & $\phi,\varphi$ &  chi & $X$ & $\chi$ \\
  psi & $\Psi$ & $\psi$  &  omega & $\Omega$ &$\omega$ \\
\hline
\end{tabular}


\newpage

\section{An introduction to sampling via measure transport}
Given a transport map, one can generate arbitrarilu many
independent and unweighted samples from
the target simply by pushig forward reference
samples through the map.

Imagine generating samples $x_i$ that are
distributed according to $u_{ref}$ and then applying
$T$ to each of these samples. Then the
transformed samples $T(x_i)$ are distributed
according to $u_{tar}$


\begin{itemize}
\item constructing transport maps given the
ability to evaluate only the unnormalized
probability density of the target
\item constructing transport maps given only
samples from a distribution of interest, but
no explicit density.
\end{itemize}

\subsection{Transport maps and optimal transport}
Let $u_{tar}\in \mathcal{B}(R^n) \rightarrow R_{+}$
be a probability measure that we wish to
characterize, defined over the Borel-$\sigma$-algebra on $R^n$.
Let $u_{ref}\in \mathcal{B}(R^n) \rightarrow R_{+}$
be another probability measure from which we gen easily generate independent and unweighted samples,e.g. a standard Gaussian. Then a transport map $T: R^n \rightarrow R^n$ pushes forward $u_{ref}$ to $u_{tar}$ if and only if $u_{tar}(A) = u_{ref}(T^{-1}(A))$ for any set $A \in \mathcal{B}(R^n)$. We can write this compactly as:
\begin{equation}\label{eq:tansportmapdefine}
T_{\sharp}u_{ref} = u_{tar}
\end{equation}
A transport map $T$ satifying (\ref{eq:tansportmapdefine}) can be understood as a deterministic coupling of two probability measures.

There may be infinitely many such transormations, however. One way of choosing a particular map is to introduce a transport cost $c:R^n \times R^n \rightarrow R$ such that $c(x,z)$ represents the work needed to move a unit of mass from $x$ to $z$. The resulting cost of a particular map is then
\begin{equation}\label{eq:mapcostdefine}
C(T) = \int c(x,T(x))du_{ref}(x)
\end{equation}
Minimizing (\ref{eq:mapcostdefine}) while simutaneously satisfying (\ref{eq:tansportmapdefine}) corresponds to a problem first posed by Mone in 1781. The solution of this constrained minimization problem is the \emph{optimal transport map}.

\subsection{Direct transport: constructing maps from unnormalized densities}
In this subsection we show how to construct a transport map that pushes forward a reference measure to the target measure when only evaluations of the \emph{unnormalized target density} are available.
\subsubsection{Preliminaries}
We assume that both target and reference measures are absolutely continuous with respect to the Lebesgue measure on $R^n$. Let $\pi$ and $\eta$ be, respectively, the normalized target and reference densities with respect to the Lebesgue measure. We seek a diffeomorphism $T$(a smooth function with smooth inverse) that pushes forward the reference to the target measure,
\begin{equation}\label{eq:differomorhpism}
u_{tar} = u_{ref} \circ T^{-1}
\end{equation}
where $\circ$ denotes the composition of functions. In terms of densities, we will rewrite (\ref{eq:differomorhpism}) as $\pi = T_{\sharp}\eta $.  $T_{\sharp}\eta$ is the pushforward of the reference density under the map $T$, and it is defined as:
\begin{equation}
T_{\sharp}\eta := \eta \circ T^{-1} |det \nabla T^{-1}|
\end{equation}
where $\nabla T^{-1}$ denotes the Jacobian of the inverse of the map.

\textbf{if $(x_i)_i$ are independent samples from $\eta$, then $(T(x_i))_i$ are independnt samples from $T_{\sharp}\eta$}. Hence, if we find a transport map $T$ that satisfies $T_{\sharp}\eta = \pi$, then then $(T(x_i))_i$ will be independent samples from the target distribution. In particular, the change of variables formula:
\begin{equation}
\int g(x)\pi(x) dx = \int[g \circ T] (x) \eta(x)dx
\end{equation}
The map therefore allows for direct computation of posterior expectations.
\subsubsection{Optimization problems}
Let $\Upsilon$ be an appropriate set of diffeomorhpisms. Then, any global minimizer of the optimization problem:
\begin{equation}\label{eq:minimizeropti}
\begin{aligned}
min \quad D_{KL}(T_{\sharp}\eta || \pi)\\
s.t. \quad det \nabla T > 0 \\
T \in \Upsilon
\end{aligned}
\end{equation}
In fact, any global minimizer of (\ref{eq:minimizeropti}) achieves the minimum cost $D_{KL}(T_{\sharp}\eta || \pi)=0$ and implies that $T_{\sharp}\eta = \pi$. The constraint det $\nabla T >0$ ensures that the pushforward density $T_{\sharp}\eta$ is strictly postive on the support of the target.

AAmong these minimizers, a particularly useful map is given by the Knothe-Rosenblatt rearrangement.

\emph{Carlier, G., Galichon, A., Santambrogio, F.: From Knothe’s transport to Brenier’s map and a continuation
method for optimal transport. SIAM Journal on Mathematical Analysis 41(6), 2554–2576 (2010)
}

We can furthe constrain (\ref{eq:minimizeropti}) so that the Knothe-Rosenblatt rearrangement is the unique global minimizer of:
\begin{equation}\label{eq:minimizeroptiunique}
\begin{aligned}
min \quad D_{KL}(T_{\sharp}\eta || \pi)\\
s.t. \quad det \nabla T \succ 0 \\
T \in \Upsilon_{\Delta}
\end{aligned}
\end{equation}
where $\Upsilon_{\Delta}$ is now the vector of smooth traingular maps. The constraint $det \nabla T \succ 0$ suffices to enforce invertibility of a feasible triangular map.

Let $\bar{\pi}$ denote any inormailized version of the target density. For any map $T$ in the feasible set of \eqref{eq:minimizeroptiunique}, the object function can be written as:
\begin{equation}
D_{KL}(T_{\sharp}\eta || \pi) = D_{KL}(\eta || T_{\sharp}^{-1}\pi ) = \mathbb{E}[-log \bar{\pi} \circ T - log det \nabla T] + \mathfrak{E}
\end{equation}
$\mathfrak{E}$ is a term independent of the transport map and thus a constant for the purposes of optimization. The resulting optimization problem reads as:

\begin{equation}\label{eq:intergrationoptimization}
\begin{aligned}
min \quad \mathbb{E}[-log \bar{\pi} \circ T - log det \nabla T]\\
s.t. \quad det \nabla T \succ 0 \\
T \in \Upsilon_{\Delta}
\end{aligned}
\end{equation}
Notice that we can evaluate the objective of \eqref{eq:intergrationoptimization} given only the unnormalized density $\bar{\pi}$ and a way to compute the integral $\mathbb{E}_{\eta}[\cdot]$. There exist a host of techniques to approximate the integral with respect to the reference measure, including quadrature and cubature formulas, sparse quadratures, Monte Carlo methods, and quasi-Monte Carlo(QMC) methods.

\eqref{eq:intergrationoptimization} is a linearly constrained nonlinear differentiable optimization problem.

\subsubsection{Convergence, bias, and approximate maps}
A transport map provides a deterministic solution to the problem of sampling from a given unnormalized density, avoiding classical stochastic tools such as MCMC. A major concern in MCMC sampling methods is the lack of clear and generally applicable convergence criteria. In the transport map framework, on the other hand, the convergence criterion is borrowed directly from standard optimization theory.



The KL divergence $D_{KL}(T_{\sharp}\eta || \pi)$ can be estimated as
\begin{equation}
D_{KL}(T_{\sharp}\eta || \pi) \approx \frac{1}{2} \mathbb{V}ar_{\eta} [log \eta - log T_{\sharp}^{-1} \pi]
\end{equation}

...... Too long.......


\subsection{Inverse transport: constructing maps from samples}
In this section, we assume that the target density is unknown and that we are only given a finite number of samples distributed according to the target measure. We show that under these hypotheses it is possible to efficiently compute an  \emph{ inverse transport}-- a transport map that pushes forward the target to the reference measure -- via convex optimization.
\subsubsection{Optimization problem}
The inverse transport pushes forward the target to the reference measure:
\[
\mu_{\ref} = \mu_{tar} \circ S^{-1}
\]










\section{Inference via low-dimensional couplings}
The transport map $T$ can be viewed as a transormation that moves particles : given a collection of samples from $v_{\eta}$ , $T$ rearranges them in accordance with the new distribution $v_{\pi}$

Optimal transprt maps, for instance, define couplings that minimize a particular integrated transport cost expressing the effort required to rearrage samples. In recent years, several other couplings have been proposed for use in statistical problems, e.g.,

\begin{itemize}
\item parametric approximations -- Moselhy, T. and Marzouk, Y. (2012). Bayesian inference with optimal maps. Journal of Computational Physics 231 7815–7850.
\item Knote-Rosenblatt rearrangement-- Rosenblatt, M. (1952). Remarks on a multivariate transformation. The Annals of Mathematical Statistics 470–472
\item coupling induced by ODE flows--- Heng, J., Doucet, A. and Pokern, Y. (2015). Gibbs flow for approximate transport with applications to Bayesian computation. arXiv:1509.08787.
Daum, F. and Huang, J. (2008). Particle flow for nonlinear filters with log-homotopy. In SPIE Defense and Security Symposium 696918–696918. International Society for Optics and Photonics.
Anderes, E. and Coram, M. (2012). A general spline representation for nonparametric and semiparametric density estimates using diffeomorphisms. arXiv:1205.5314.
\end{itemize}

\textbf{Yet the construction, representation, and evaluation of all these maps grows challenging in high dimensions.}

The central contribution of this paper is to \textbf{establish a link between the conditional independence structure of the target measure and the existence of special low dimensional coupling. These couplings are induced by transport maps that are sparse or decomposable.}

\begin{itemize}
\item sparse: A sparse map consists of scalar-valued compnent functions that each depend only a few input variables
\item decomposable map: A decomposable map factorizes as the exact composition of finitely many functions of low effective dimension(i.e, $T = T_1 \circ ... \circ T_l$, where each $T_i$ differs from the identity map only along a subset of its components).
\end{itemize}

The utility of thses results is twofold:
\begin{itemize}
\item First, they make the construction of couplings tractable for a large class of inference problems.
\item Second, they suggest new algorithmic approaches for important classed of statistical models.
\end{itemize}

\subsection{Notation}
\begin{itemize}
\item $f \circ g$: the composition of $f,g$
\item $\partial_k f$ partial derivative of $f$ with respect to its $k$th input variable
\item $x \rightarrow Qx$: linear map
\item $T_{\sharp} v$: pushforward measure given by $v \circ T^{-1}$
\item $T^{-1}(B)$: set-valued preimage of $B$ under T
\item $T^{\sharp}v$: the pullback measure given by $v \circ T$
\item $T_{\sharp} \pi$:
\end{itemize}

\subsection{Triangular transport maps: a building block}





\section{An Optimal Transport Formulation of the Linear Feedback Particle
Filter}
Based on the concept of optimal transportation, a time-stepping optimization procedure is proposed here to obtain a unique optimal control law, denoted as $\mu_t^*$ and $K_t^*$. In this procedure, a finite time interval is divided into discrete time steps $t_0,...,t_n$.
Then a discrete time random process, $S_{t_k}$ is constructed by initializing $S_{t_0}$ according to the initial prior $P(X_0)$, and sequentially evolving $S_{t_k} \rightarrow S_{t_{k+1}}$ at each time-step with a map denoted by $T_k$:
\begin{equation}
S_{t_{k+1}} = T_k(S_{t_k}),  S_0   P(X_0)
\end{equation}

% The map $T_k$ is obtained by solving an optimal transportation problem between the conditional probability distributions at time instants $t_{k}$ and $t_{k+1}$, between $P(X_{t_k}|Z_{t_k})$ and $P(X_{t_{k+1}}|Z_{t_{k+1}})$.


% By construction, $S_{t_k}$ is distributed according to $P(X_{t_k}|Z_{t_k}).


\subsection{Exactness and non-uniqueness}




\section{transportmaps.mit.edu}
\subsection{Bayesian models}
Bayesian models arise naturally in statistical inference problems, where the belief represented by a prior probability distribution
need to be updated according to some observations. We can summarize these models as follows.

Let $G: \mathbb{R}^d \rightarrow \mathbb{R}^{d_z}$ be a map, let $X \in \mathbb{R}^d$ be a random variable with distribution $v_{\rho}$, representing the prior belief on the state of $X$. Let $d \in \mathbb{R}^{d_y}$ be observations of the output of $G$ for some unknown inputs $x \in \mathbb{R}^d$ obtained through the observations. This model for the measurements
\begin{equation}
d = f(G(x),\xi)
\end{equation}
we defin the posterior distribution $v_{\pi}$ by its density
\[
 \pi(x|Y = d) \approx \mu(f^{-1}(G(x),d)) \rho(x) = \mathcal{L}(x) \rho(x)
 \]
 $ \mathcal{L}(x)  $ is the likelihood and $  \rho(x)$ is the prior density.
 \subsubsection{Linear Gaussian model}
 Let us consider here the linear model $G(x) = Gx$ with additive Gaussian noise
 \[
 d = G x + \xi
 \]
 This means that the likelihood is defined as
 \[
\mathcal{L}_d(x) = \mu(d - Gx)
 \]

Let the prior distribution on $X$ be Gaussian, $v_{\rho} \sim \mathcal{N}(m_{\rho},\Sigma_{\rho})$.

\textbf{Feedback particle filter:} The feedback particle filter for linear Gaussian problem is given by


\subsection{Background on optimal transportation}
Let $P_X$ and $P_Y$ be two given probability measures on $\mathbb{R}^d$ with finite second moments. The optimal transportation problem is to minimize
\[
 min_T \quad E[(T(x) - x)^2]
\]
over all maps $T: \mathbb{R}^d \rightarrow \mathbb{R}^d$ such that:
\[
X \sim P_X, \quad T(x) \sim P_Y
\]
If it exists, the minimizer $T^*$, is called the optimal transport map between $P_X$ and $P_Y$. The optimal cost is referred to as $L^2-$Wasserstein distance between $P_X$ and $P_Y$.

\subsubsection{Optimal map between Gaussians}
Suppose $P_X$ and $P_Y$ are Gaussian distributions, $\mathcal{N}(X,\Sigma_X)$ and $\mathcal{N}(Y,\Sigma_Y)$. Then the optimal transport map between $P_X$ and $P_Y$ is given by:
\[
T(x) = Y + F(x - X)
\]
where
\[
F = \Sigma_Y^{\frac{1}{2}}(\Sigma_Y^{\frac{1}{2}} \Sigma_X \Sigma_Y^{\frac{1}{2}})^{-\frac{1}{2}}\Sigma_Y^{\frac{1}{2}}
\]




\section{Classical Monge-Kantorovich problem}
In our simplest exercise we generate two of these random histograms and then try to design a transport plan between them. The concise statement of the problem is
\[
min \left\{Tr M^T Z|Z1 = p,Z^T1 = q \right\}
\]
where $p,q$ are $m$ vectors representing the mass of the histogram bars for the source and destination histograms, respectively. The matrix $M$ represents the distances between the pairs of historgram bars, i.e. $M_{i,j} = |x_i - y_j|$. The trace formulation of the objective is simply the discrete analogue of the continuous formulation
\[
min_{\pi \in \prod(\mu,v)} \int c(x,y)d\pi(x,y)
\]
where $\pi(\mu,v)$ for all borel sets $A$ and $B$.  The matrix $Z$ in the discretized problem playes the role of $\pi$ in the continuous formulation, indicating how much mass is to be transfered from points $x$ to point $y$.

\section{Optimal Transport Filtering with Particle Reweighing in Finance}

Particle flow filter(daum) allows the reduction of the number of particles we need in order to get a tolerable level of errors in the filtering problem. The main idea behind this method is the evolution in homotopy parameter $\lambda$ from prior to the target density. The introduced a particle flow, in which particles are gradually transported without the necessity to randomly sample from any distribution.

The idea of transportation and rewighing mechanism is to transport particles through the sequence of densities that move the least during the synthetic time until they reach the posterior distribution. By regenerating particles according  to their weight at each time step we are able to direct the flow and further minimize the variance of the estimates.



\end{CJK}
\end{document}